\documentclass[noborder,onecolumn]{netmap}
\usepackage[absolute,overlay]{textpos}
\usepackage{setspace}
\usepackage{wwfont}
\begin{document}

\epsfigure[80]{west-coast-detail.eps}{}
\begin{textblock}{32}(12,44)
  \textit{Drawn by: William Waites (wwaites@tardis.ed.ac.uk), October 2012}
\end{textblock}
\begin{textblock}{32}(12,4)
  \textbf{\Huge Network Map of Hebnet, Knoydart and Tegola}
\end{textblock}
\begin{textblock}{10}(74,6)
  \linespread{1.0}
  \scriptsize
  The first iteration of the Tegola network began in 2008 and was
    constructed as a ring around Loch Hourn. It is made with custom
    hardware based on the Gateworks Cambria series single board computer
    running the OpenWRT flavour of Linux. Each inter-site connection is
    actually a pair of long-haul links with orthogonal polarisation. Until
    the commissioning of the production network in Loch Hourn in the
    summer of 2012, links polarised in one way carried user traffic and
    links polarised in the other way were used for experiments. The
    interior gateway protocol is OSPF.

    Partly to take advantage of a significant drop in price of
    wireless equipment, and partly because the community had become
    reliant on the network and on occasion experiments
    interfered with live traffic, a parallel network was built in the
    summer of 2012. The new network shares some physical space with the
    old, and maintains out-of-band links for instrumentation that can also
    be used in emergencies to re-route production traffic. Most of the
    access routers are from Ubiquiti networks and run OpenWRT. Most of the
    devices used at the edge of the network are also from Ubiquiti but run
    the stock firmware.  Like the old network, the new one runs OSPF for
    dynamic routing within the network.
\end{textblock}
\begin{textblock}{10}(70,28)
  \linespread{1.0}
  \scriptsize
  The network covering the South coast of the Knoydart peninsula was
  built using Ubiquiti equipment and the architecture is based on that
  of Rùm and Eigg. The routers run the stock firmware and IP routing
  rather than bridging is used. Static routes are used and due to the
  size of the network, the routing tables are pre-calculated using a
  spreadsheet. This works because the networks is tree-shaped and so
  contains no cycles or loops. OSPF is used between Mallaig and the
  backbone site on the Creagan Dearga to facilitate fail-over.
\end{textblock}
\begin{textblock}{10}(4,10)
  \linespread{1.0}
  \scriptsize
  The network on R\`um was the first to use a policy-based routing
  arrangement where users were separated into groups and each group uses
  a different DSL line at Mallaig. Recently (October 2012) the network
  was extended around to Canna via Elgol on Skye and also connected to
  Eigg. It uses mostly Ubiquiti equipment and static routing. A Cisco
  2800 series router is used as the DSL gateway in Mallaig.
\end{textblock}
\begin{textblock}{10}(4,37)
  \linespread{1.0}
  \scriptsize
  The network on Eigg is built with the same Ubiquiti equipment as
  Knoydart, R\`um and Tegola but the architecture is radically
  different. Like Knoydart it is tree-shaped and loop-free, and it is
  quite a large tree indeed. It also employs no routing internally,
  the entire network being one large bridge. Furthermore network
  address translation is used at the edge node so that the backbone
  does not need to know anything about the edge subnets. This works
  quite well except for problems with some application protocols 
  such as SIP that are NAT-unfriendly. The exception, as with
  Knoydart, is that IP routing and OSPF is used on the connection
  between the backbone router on the Sgurr of Eigg and the DSL lines
  in Arisaig.  
\end{textblock}
\begin{textblock}{10}(41,10)
  \linespread{1.0}
  \scriptsize
  The community networks benefit from a high quality connection to the
  Ja.net academic backbone through facilities made available by Sabhal
  Mòr Ostaig, the UHI Gaelic college on Skye.
\end{textblock}
\begin{textblock}{10}(40,20)
  \linespread{1.0}
  \scriptsize
  As of October 2012, the networks on Knoydart, Eigg, and the two
  Tegloa networks peer with each other using BGP, in exactly the same
  way that the networks that make up the Internet connect. This means
  that each is able to maintain its own architecture and policies
  internally but nevertheless can have a consistent macroscopic view
  of this network of networks. Mutual ``transit'' means that each
  network can share its upstream connection with the others for
  fail-over and resiliency.
\end{textblock}
\begin{textblock}{10}(54,41)
  \linespread{1.0}
  \scriptsize
  At various locations in Mallaig and Arisaig the individual networks
  maintain connections to the Internet using the only alternative that
  is readily and cheaply available -- DSL lines. This generally works
  but can be problematic since it is a retail consumer product and
  neither designed nor intended as an interconnection medium for
  telecommunications carriers.
\end{textblock}

\end{document}
