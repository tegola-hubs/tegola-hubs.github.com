\documentclass[tocflat]{article}
\usepackage[a4paper,top=1in,bottom=1in,right=1in,left=1.25in]{geometry}
\usepackage{enumerate}
\usepackage{tocstyle}
\usepackage{marginnote}
\usepackage{fontspec}
\defaultfontfeatures{Mapping=tex-text}
\setmainfont[
  UprightFeatures    = {SmallCapsFont={Alegreya SC}},
  BoldFeatures       = {SmallCapsFont={Alegreya SC Bold}},
  ItalicFeatures     = {SmallCapsFont={Alegreya SC Italic}},
  BoldItalicFeatures = {SmallCapsFont={Alegreya SC Bold Italic}}
]{Alegreya}
\setsansfont{Cabin}
\newfontfamily\sfbf{Cabin Bold}
\newcommand{\textsfbf}[1]{{\sfbf #1}}

\newcommand{\companyname}{High-speed Universal Broadband Scotland
  \textsc{c.i.c}}%
\newcommand{\object}{improve the availability and quality of Internet
  service in rural scotland by supporting community networking
  projects through aggregation of demand for networking products
  services, acting in the marketplace on behalf of its Members, and
  providing such assistance, training and support to its Members as
  may be necessary}
\newcommand{\nominee}{University of Edinburgh}
\newcommand{\nomineecharity}{SC005336}
\newcommand{\nomineecompany}{N/A}
\newcommand{\nomineeaddress}{Old College\\
South Bridge\\
Edinburgh\\
Scotland\\
EH8 9YL}
\newcommand{\quorum}{two}%

\renewcommand*{\raggedleftmarginnote}{}{}%
\setlength{\marginparsep}{-1em}%
\reversemarginpar

\renewcommand{\contentsname}{\textsc{Index to the Articles}}%

\newcounter{clause}%
\newcounter{subclause}[clause]%
\newcounter{schedule}%

\newcommand{\heading}[1]{%
  \vspace{2\baselineskip}%
  \addcontentsline{toc}{section}{#1}%
  \begin{center}%
    \textsc{\large #1}\\%
  \end{center}%
}%

\newcommand{\addtocschedclause}[1]{%
  \addcontentsline{toc}{section}{%
    \protect\numberline{\theclause} \#1%
  }%
}%

\newcommand{\schedule}[1]{%
  \vspace{2\baselineskip}%
  \setcounter{clause}{0}%
  \refstepcounter{schedule}%
  \renewcommand{\theclause}{S\arabic{schedule}.\arabic{clause}}%
  \renewcommand{\thesubclause}{S\arabic{schedule}.\arabic{clause}.\arabic{subclause}}%
  \renewcommand{\theschedule}{S\arabic{schedule}}%
  \addcontentsline{toc}{section}{\protect\numberline{S\arabic{schedule}}%
    Schedule -- #1}%
  \begin{center}%
    \textsc{\large Schedule \arabic{schedule} -- #1}\\%
  \end{center}%
}%

\newenvironment{clause}[1]{%
  \refstepcounter{clause}%
  \addcontentsline{toc}{section}{\protect\numberline{\theclause} #1}%
  \par\noindent\marginnote{\textbf{\theclause}}\textbf{#1}%
  \par\noindent%
}{\vspace{\baselineskip}}%

\newenvironment{clause*}{%
  \refstepcounter{clause}%
  \par\vspace{0.5\baselineskip}\par\noindent%
  \marginnote{\textbf{\theclause}}%
}{\vspace{\baselineskip}}%

\newenvironment{subclause}{%
  \refstepcounter{subclause}%
  \par\vspace{0.5\baselineskip}\par\noindent%
  \marginnote{\thesubclause}%
}{\par}%

\renewcommand{\theclause}{\arabic{clause}}%
\renewcommand{\thesubclause}{\arabic{clause}.\arabic{subclause}}%
\renewcommand{\theschedule}{}%

\newenvironment{terms}{\begin{description}}{\end{description}}
\newcommand{\term}[2]{\item[#1] #2}

\begin{document}%

\thispagestyle{empty}%
\begin{center}%
  \textsc{The Companies Act 2006}\\
  \vspace{2em}%
  \textsc{Community Interest Company Limited by Guarantee}\\
  \vspace{8em}%
  \hrule
  \vspace{2em}%
  \textsc{Articles of Association}\\
  \vspace{1em}%
  of\\
  \vspace{1em}%
  \companyname\\
  \vspace{2em}%
  \hrule
\end{center}%
\newpage

\begin{center}%
  The Companies Act 2006\\
  Community Interest Company Limited by Guarantee\\
\end{center}%
\vspace{2em}%

\tableofcontents
\newpage
%
\begin{center}%
  The Companies Act 2006\\
  Articles of Association\\
  of\\
  \companyname\\
\end{center}%
\vspace{2em}%
%
\heading{Interpretation}%
%
\begin{clause}{Defined Terms}%
  The interpretation of these Articles is governed by the provisions
  set out in the Schedule at end of the Articles.
\end{clause}%
\heading{Community and Interest Company and Asset Lock}%
\begin{clause}{Community Interest Company}%
  The Company is to be a community interest company.
\end{clause}%
\begin{clause}{Asset Lock}%
  \begin{subclause}%
    \label{cl:notransfer}%
    The Company shall not transfer any of its assets other than
    for full consideration.
  \end{subclause}%
  \begin{subclause}%
    \label{cl:xfr-assetlock}%
    Provided the conditions in Article \ref{cl:xfr-consideration} are
    satisfied, Article \ref{cl:notransfer} 
    shall not apply to:
    \begin{enumerate}[(a)]%
      \item the transfer of assets to any specified asset-locked body,
        or (with the consent of the Regulator) to any other
        asset-locked body; and
      \item the transfer of assets made for the benefit of the
        community other than by way of a transfer of assets into an
        asset-locked body.
    \end{enumerate}%
  \end{subclause}%
  \begin{subclause}%
    \label{cl:xfr-consideration}%
    The conditions are that the transfer of assets must comply with
    any restrictions on the transfer of assets for less than full
    consideration which may be set out elsewhere in the Memorandum and
    Articles of the Company.
  \end{subclause}%
  \begin{subclause}%
    \label{cl:xfr-windup}%
    If:
    \begin{enumerate}[(a)]%
      \item the Company is wound up under the Insolvency Act 1986; and
      \item all its liabilities have been satisfied
    \end{enumerate}%
    any residual assets shall be given or transferred to the
    asset-locked body specified in Article \ref{cl:xfr-nominee} below.
  \end{subclause}%
  \begin{subclause}%
    \label{cl:xfr-nominee}%
    For the purposes of this Article \arabic{clause}, the following
    asset-locked body is specified as a potential recipient of the
    Company’s assets under Articles \ref{cl:xfr-assetlock} and
    \ref{cl:xfr-windup}: 
    \par\vspace{\baselineskip}\par%
    \begin{tabular}{p{0.5\textwidth}p{0.5\textwidth}}%
      Name: & \nominee\\
      \vspace{0.5\baselineskip} & \\
      Registered Office: &
      \begin{minipage}{0.5\textwidth}%
        \nomineeaddress%
      \end{minipage}\\%
      \vspace{0.5\baselineskip} & \\
      Charity Registration Number: & \nomineecharity\\
      Company Registration Number: & \nomineecompany\\
    \end{tabular}%
  \end{subclause}%
\end{clause}%
%
\begin{clause}{Not for profit}%
  The Company is not established or conducted for private gain: any
  surplus or assets are used principally for the benefit of the
  community.
\end{clause}%
%
\heading{Objects, Powers and Limitation of Liability}%
%
\begin{clause}{Objects}%
  The objects of the Company are to carry on  activities  which  benefit
  the  community  and  in   particular   (without   limitation)   to
  \object. 
\end{clause}%
\begin{clause}{Powers}%
  To further its objects the Company may do all such lawful things as
  may further the Company’s objects and, in particular, but, without
  limitation, may borrow or raise and secure the payment of money for
  any purpose including for the purposes of investment or of raising
  funds.
\end{clause}%
\begin{clause}{Liability of members}%
  The liability of each member is limited to £1, being the amount that
  each member undertakes to contribute to the assets of the Company in
  the event of its being wound up while he or she is a member or
  within one year after he or she ceases to be a member, for:
  \begin{enumerate}[(a)]%
    \item payment of the Company’s debts and liabilities contracted before he
      or she ceases to be a member;
    \item payment of the costs, charges and expenses of winding up; and
    \item adjustment of the rights of the contributories among
      themselves.
  \end{enumerate}%
\end{clause}%
%
\newpage%
\heading{Directors}%
\heading{Directors' Powers and Responsibilities}%
%
\begin{clause}{Directors’ general authority}%
  Subject to the Articles, the Directors are responsible for the
  management of the Company’s business, for which purpose they may
  exercise all the powers of the Company.
\end{clause}%
\begin{clause}{Members’ reserve power}%
\begin{subclause}%
  The members may, by special resolution, direct the Directors to  take,
  or refrain from taking, specific action.
\end{subclause}%
\begin{subclause}%
  No such special resolution invalidates anything which the Directors
  have done before the passing of the resolution.
\end{subclause}%
\end{clause}%
\begin{clause}{Chair}%
  \label{cl:chair}%
  The Directors may appoint one of their number to be the chair of the
  Directors for such term of office as they determine and may at any
  time remove him or her from office.
\end{clause}%
\begin{clause}{Directors may delegate}%
  \begin{subclause}%
    Subject to the Articles, the Directors may delegate any of the  powers
    which are conferred on them under the Articles:
    \begin{enumerate}[(a)]%
      \item to such person or committee;
      \item by such means (including by power of attorney);
      \item to such an extent;
      \item in relation to such matters or territories; and
      \item on such terms and conditions;
    \end{enumerate}%
    as they think fit.
\end{subclause}%
\begin{subclause}%
  If the Directors so specify, any such delegation may authorise
  further delegation of the Directors’ powers by any person to whom
  they are delegated.
\end{subclause}%
\begin{subclause}%
  The Directors may revoke any delegation in whole or part, or alter
  its terms and conditions.
\end{subclause}%
\end{clause}%
\begin{clause}{Committees}%
  \begin{subclause}%
    Committees to which the Directors delegate any of their powers
    must follow procedures which are based as far as they are
    applicable on those provisions of the Articles which govern the
    taking of decisions by Directors.
  \end{subclause}%
  \begin{subclause}%
    The Directors may make rules of procedure for all or any
    committees, which prevail over rules derived from the Articles if
    they are not consistent with them.
  \end{subclause}%
\end{clause}%
%
\heading{Decision-making by Directors}%
%
\begin{clause}{Directors to take decisions collectively}%
  Any decision of the Directors must be either a majority decision at  a
  meeting or a decision taken in accordance with Article
  \ref{cl:decision-nomeeting}.
\end{clause}%
\begin{clause}{Calling a Directors’ meeting}%
  \begin{subclause}%
    Two Directors may (and the Secretary, if any, must at the  request  of
    two Directors) call a Directors’ meeting.
  \end{subclause}%
  \begin{subclause}%
    A Directors’ meeting must be called by  at  least  seven  Clear  Days’
    notice unless either:
    \begin{enumerate}[(a)]%
      \item all the Directors agree; or
      \item urgent circumstances require shorter notice.
    \end{enumerate}%
  \end{subclause}%
  \begin{subclause}%
    Notice of Directors’ meetings must be given to each Director.
  \end{subclause}%
  \begin{subclause}% 
    Every notice calling a Directors’ meeting must specify:
    \begin{enumerate}[(a)]%
      \item the place, day and time of the meeting; and
      \item if it is anticipated that Directors participating in the meeting
        will not be in the same place, how  it  is  proposed  that  they
        should communicate with each other during the meeting.
    \end{enumerate}%
  \end{subclause}% 
  \begin{subclause}%
    Notice of Directors’ meetings need not be in Writing.
  \end{subclause}%
  \begin{subclause}%
    Notice of Directors’ meetings may be sent by Electronic  Means  to  an
    Address provided by the Director for the purpose.
  \end{subclause}%
\end{clause}%
\begin{clause}{Participation in Directors’ meetings}%
  \label{cl:participation}
  \begin{subclause}%
    Subject  to  the  Articles,  Directors  participate  in  a  Directors’
    meeting, or part of a Directors’ meeting, when:
    \begin{enumerate}[(a)]%
      \item the meeting has been called and takes place in accordance
        with the Articles; and
      \item they can each communicate to the others any information or
        opinions they have on any particular item of the business of the
        meeting.
    \end{enumerate}%
  \end{subclause}%
  \begin{subclause}%
    In determining whether Directors are  participating  in  a  Directors’
    meeting,  it  is  irrelevant  where  any  Director  is  or  how   they
    communicate with each other.
  \end{subclause}%
  \begin{subclause}%
    If all the Directors participating in a meeting are not in the
    same place, they may decide that the meeting is to be treated as
    taking place wherever any of them is.
  \end{subclause}%
\end{clause}%
\begin{clause}{Quorum for Directors’ meetings}%
  \begin{subclause}%
    At a Directors’ meeting, unless a quorum is participating, no
    proposal is to be voted on, except a proposal to call another
    meeting.
  \end{subclause}%
  \begin{subclause}%
    The quorum for Directors’ meetings may be fixed from time to time
    by a decision of the Directors, but it must never be less than
    two, and unless otherwise fixed it is \quorum.
  \end{subclause}%
  \begin{subclause}%
    If the total number of Directors for the time being is less than
    the quorum required, the Directors must not take any decision
    other than a decision:
    \begin{enumerate}[(a)]%
      \item to appoint further Directors; or
      \item to call a general meeting so as to enable the members to
        appoint further Directors.
    \end{enumerate}%
  \end{subclause}%
\end{clause}%
\begin{clause}{Chairing of Directors’ meetings}%
  The Chair, if any, or in his or her absence another Director
  nominated by the Directors present shall preside as chair of each
  Directors’ meeting.
\end{clause}%
\begin{clause}{Decision making at a meeting}%
  \begin{subclause}%
    Questions arising at a  Directors’  meeting  shall  be  decided  by  a
    majority of votes.
  \end{subclause}%
  \begin{subclause}%
    In all proceedings of Directors each Director must not have more  than
    one vote.
  \end{subclause}%
  \begin{subclause}%
    In case of an equality of votes, the Chair  shall  have  a  second  or
    casting vote.
  \end{subclause}%
\end{clause}%
\begin{clause}{Decisions without a meeting}%
  \label{cl:decision-nomeeting}%
  \begin{subclause}%
    \label{cl:decision-commonview}%
    The Directors may take a unanimous decision without a Directors’
    meeting by indicating to each other by any means, including
    without limitation by Electronic Means, that they share a common
    view on a matter.  Such a decision may, but need not, take the
    form of a resolution in Writing, copies of which have been signed
    by each Director or to which each Director has otherwise indicated
    agreement in Writing.
  \end{subclause}%
  \begin{subclause}%
    \label{cl:decision-valid}%
    A decision which is made in accordance with Article
    \ref{cl:decision-commonview} shall be as valid and effectual as if
    it had been passed at a meeting duly convened and held, provided
    the following conditions are complied with:
    \begin{enumerate}[(a)]%
      \item approval from each Director must be received by one person
        being either such person as all the Directors have nominated
        in advance for that purpose or such other person as volunteers
        if necessary (“the Recipient”), which person may, for the
        avoidance of doubt, be one of the Directors;
      \item following receipt of responses from all of the Directors,
        the Recipient must communicate to all of the Directors by any
        means whether the resolution has been formally approved by the
        Directors in accordance with this Article
        \ref{cl:decision-valid};
      \item the date of the decision shall be the date of the
        communication from the Recipient confirming formal approval;
      \item the Recipient must prepare a minute of the decision in
        accordance with Article \ref{cl:minutes}.
    \end{enumerate}%
  \end{subclause}%
\end{clause}%
\begin{clause}{Conflicts of interest}%
  \begin{subclause}%
    Whenever a Director finds himself or herself in a situation that
    is reasonably likely to give rise to a Conflict of Interest, he or
    she must declare his or her interest to the Directors unless, or
    except to the extent that, the other Directors are or ought
    reasonably to be aware of it already.
  \end{subclause}%
  \begin{subclause}%
    If any question arises as to whether a Director has a Conflict of
    Interest, the question shall be decided by a majority decision of
    the other Directors.
  \end{subclause}%
  \begin{subclause}%
    \label{cl:conflict-withdraw}%
    Whenever a matter is to be discussed at a meeting or decided in
    accordance with Article \ref{cl:decision-nomeeting} and a Director
    has a Conflict of Interest in respect of that matter then, subject
    to Article \ref{cl:conflict-auth}, he or she must:
    \begin{enumerate}[(a)]%
      \item remain only for such part of the meeting as in the view of
        the other Directors is necessary to inform the debate;
      \item not be counted in the quorum for that part of the meeting;
        and
      \item withdraw during the vote and have no vote on the matter.
    \end{enumerate}%
  \end{subclause}%
  \begin{subclause}%
    When a Director has a Conflict of Interest which he or she has
    declared to the Directors, he or she shall not be in breach of his
    or her duties to the Company by withholding confidential
    information from the Company if to disclose it would result in a
    breach of any other duty or obligation of confidence owed by him
    or her.
  \end{subclause}%
\end{clause}%
\begin{clause}{Directors’ power to authorise a conflict of interest}%
  \label{cl:conflict-auth}%
  \begin{subclause}%
    \label{cl:conflict-authpower}%
    The Directors have power to authorise a Director to be in  a  position
    of Conflict of Interest provided:
    \begin{enumerate}[(a)]%
    \item in relation to the decision to authorise a Conflict of Interest,
      the conflicted Director must comply with Article
      \ref{cl:conflict-withdraw};
    \item in authorising a Conflict of Interest, the Directors can
      decide the manner in which the Conflict of Interest may be dealt
      with and, for the avoidance of doubt, they can decide that the
      Director with a Conflict of Interest can participate in a vote
      on the matter and can be counted in the quorum;
    \item the decision to authorise a Conflict of Interest can impose
      such terms as the Directors think fit and is subject always to
      their right to vary or terminate the authorisation; and
    \end{enumerate}%
  \end{subclause}%
  \begin{subclause}%
    If a matter, or office, employment or position, has been
    authorised by the Directors in accordance with Article
    \ref{cl:conflict-authpower} then, even if he or she has been
    authorised to remain at the meeting by the other Directors, the
    Director may absent himself or herself from meetings of the
    Directors at which anything relating to that matter, or that
    office, employment or position, will or may be discussed.
  \end{subclause}%
  \begin{subclause}%
    A Director shall not be accountable to the Company for any benefit
    which he or she derives from any matter, or from any office,
    employment or position, which has been authorised by the Directors
    in accordance with Article \ref{cl:conflict-authpower} (subject to
    any limits or conditions to which such approval was subject).
  \end{subclause}%
\end{clause}%
\begin{clause}{Register of Directors’ interests}%
  The Directors shall cause a register of Directors’ interests to be
  kept.  A Director must declare the nature and extent of any
  interest, direct or indirect, which he or she has in a proposed
  transaction or arrangement with the Company or in any transaction or
  arrangement entered into by the Company which has not previously
  been declared.
\end{clause}%
\heading{Appointment and Retirement of Directors}%
\begin{clause}{Methods of appointing directors}%
  \begin{subclause}%
    Those persons notified to the Registrar  of  Companies  as  the  first
    Directors of the Company shall be the first Directors.
  \end{subclause}%
  \begin{subclause}%
     Any person who is willing to act as a Director, and is permitted
     by law to do so, may be appointed to be a Director:
     \begin{enumerate}[(a)]%
       \item by ordinary resolution; or
       \item by a decision of the Directors.
     \end{enumerate}%
  \end{subclause}%
  \begin{subclause}%
    In any case where, as a result of death, the Company has no
    members and no Directors, the personal representatives of the last
    member to have died have the right, by notice in writing, to
    appoint a person to be a member.
  \end{subclause}%
  \begin{subclause}%
    For the purposes of Article 23.3, where two or more members die in
    circumstances rendering it uncertain who was the last to die, a
    younger member is deemed to have survived an older member.
  \end{subclause}%
\end{clause}%
\begin{clause}{Termination of Director’s appointment}%
  A person ceases to be a Director as soon as:
  \begin{enumerate}[(a)]%
    \item that person ceases to be a Director by virtue of any
      provision of the Companies Acts, or is prohibited from being a
      Director by law;
    \item a bankruptcy order is made against that person, or an order
      is made against that person in individual insolvency proceedings
      in a jurisdiction other than England and Wales or Northern
      Ireland which have an effect similar to that of bankruptcy;
    \item  a composition is  made  with  that  person’s  creditors  generally  in
      satisfaction of that person’s debts;
    \item the Directors reasonably believe he or she is suffering from
      mental disorder and incapable of acting and they resolve that he
      or she be removed from office;
    \item notification is received by the Company from the Director
      that the Director is resigning from office, and such resignation
      has taken effect in accordance with its terms (but only if at
      least two Directors will remain in office when such resignation
      has taken effect);
    \item the Director fails to attend three consecutive meetings of
      the Directors and the Directors resolve that the Director be
      removed for this reason; or
    \item at a general meeting of the Company, a resolution is passed
      that the Director be removed from office, provided the meeting
      has invited the views of the Director concerned and considered
      the matter in the light of such views.
  \end{enumerate}%
\end{clause}%
\begin{clause}{Directors’ remuneration}%
  \begin{subclause}%
     Directors  may  undertake  any  services  for  the  Company  that  the
     Directors decide.
  \end{subclause}%
  \begin{subclause}%
    Directors are entitled to such remuneration as the Directors
    determine:
    \begin{enumerate}[(a)]%
      \item for their services to the Company as Directors; and
      \item for any other service which they undertake for the Company.
    \end{enumerate}%
  \end{subclause}%
  \begin{subclause}%
    Subject to the Articles, a Director’s remuneration may:
    \begin{enumerate}[(a)]%
      \item take any form; and
      \item include any arrangements in connection with the payment of
        a pension, allowance or gratuity, or any death, sickness or
        disability benefits, to or in respect of that director.
    \end{enumerate}%
  \end{subclause}%
  \begin{subclause}%
    Unless the Directors decide otherwise, Directors’ remuneration
    accrues from day to day.
  \end{subclause}%
  \begin{subclause}%
    Unless the Directors decide otherwise, Directors are not
    accountable to the Company for any remuneration which they receive
    as Directors or other officers or employees of the Company’s
    subsidiaries or of any other body corporate in which the Company
    is interested.
  \end{subclause}%
\end{clause}%
\begin{clause}{Directors’ expenses}%
  The Company may pay any reasonable expenses which the Directors
  properly incur in connection with their attendance at:
  \begin{enumerate}[(a)]%
    \item meetings of Directors or committees of Directors;
    \item general meetings; or
    \item separate meetings of any class of members or of the  holders  of
          any debentures of the Company,
  \end{enumerate}%
  or otherwise in connection with the exercise of their powers  and  the
  discharge of their responsibilities in relation to the Company.
\end{clause}%
\newpage%
\heading{Members}%
\heading{Becoming and Ceasing to be a Member}%
\begin{clause}{Becoming a member}%
  \begin{subclause}%
    The subscribers to  the  Memorandum  are  the  first  members  of  the
    Company.
  \end{subclause}%
  \begin{subclause}%
    Such other persons as are admitted to membership  in  accordance  with
    the Articles shall be members of the Company.
  \end{subclause}%
  \begin{subclause}%
    No person shall be admitted a member of the Company unless he or
    she is approved by the Directors.
  \end{subclause}%
  \begin{subclause}%
    Every person who wishes to  become  a  member  shall  deliver  to  the
    Company an application for membership in  such  form  (and  containing
    such information) as the Directors require and executed by him or her.
  \end{subclause}%
\end{clause}%
\begin{clause}{Termination of membership}%
  \begin{subclause}%
    Membership is not transferable to anyone else.
  \end{subclause}%
  \begin{subclause}%
    Membership is terminated if:
    \begin{enumerate}[(a)]%
      \item the member dies or ceases to exist;
      \item otherwise in accordance with the Articles; or
      \item at a meeting of the Directors at which at least half of
        the Directors are present, a resolution is passed resolving
        that the member be expelled on the ground that his or her
        continued membership is harmful to or is likely to become
        harmful to the interests of the Company.  Such a resolution
        may not be passed unless the member has been given at least 14
        Clear Days’ notice that the resolution is to be proposed,
        specifying the circumstances alleged to justify expulsion, and
        has been afforded a reasonable opportunity of being heard by
        or of making written representations to the Directors. A
        member expelled by such a resolution will nevertheless remain
        liable to pay to the Company any subscription or other sum
        owed by him or her.
    \end{enumerate}%
  \end{subclause}%
\end{clause}%
\newpage%
\heading{Organisation of General Meetings}%
\begin{clause}{General meetings}%
  \begin{subclause}%
    The Directors may call a general meeting at any time.
  \end{subclause}%
  \begin{subclause}%
    The Directors must call a general meeting if required to do so by  the
    members under the Companies Acts.
  \end{subclause}%
\end{clause}%
\begin{clause}{Length of notice}%
  All general meetings must be called by either:
  \begin{enumerate}%
    \item at least 14 Clear Days’ notice; or
    \item shorter notice if it is so agreed by a majority of the
      members having a right to attend and vote at that meeting.  Any
      such majority must together represent at least 90\% of the total
      voting rights at that meeting of all the members.
  \end{enumerate}%
\end{clause}%
\begin{clause}{Contents of notice}%
  \begin{subclause}%
    Every notice calling a general meeting must specify the place, day
    and time of the meeting, whether it is a general or an annual
    general meeting, and the general nature of the business to be
    transacted.
  \end{subclause}%
  \begin{subclause}%
    If a special resolution is to be proposed, the notice must include
    the proposed resolution and specify that it is proposed as a
    special resolution.
  \end{subclause}%
  \begin{subclause}%
    In every notice calling a meeting of the Company there must appear
    with reasonable prominence a statement informing the member of his
    or her rights to appoint another person as his or her proxy at a
    general meeting.
  \end{subclause}%
\end{clause}%
\begin{clause}{Service of notice}%
  Notice of general meetings must be  given  to  every  member,  to  the
  Directors and to the auditors of the Company.
\end{clause}%
\begin{clause}{Attendance and speaking at general meetings}%
  \begin{subclause}%
    A person is able to exercise the right to speak at a general
    meeting when that person is in a position to communicate to all
    those attending the meeting, during the meeting, any information
    or opinions which that person has on the business of the meeting.
  \end{subclause}%
  \begin{subclause}%
    A person is able to exercise the right to vote at  a  general  meeting
    when:
    \begin{enumerate}[(a)]%
      \item that person is able to vote, during the meeting, on
        resolutions put to the vote at the meeting; and
      \item that person’s vote can be taken into account in
        determining whether or not such resolutions are passed at the
        same time as the votes of all the other persons attending the
        meeting.
    \end{enumerate}%
  \end{subclause}%
  \begin{subclause}%
    The Directors may make whatever arrangements they consider
    appropriate to enable those attending a general meeting to
    exercise their rights to speak or vote at it.
  \end{subclause}%
  \begin{subclause}%
    In determining attendance at a general meeting, it is immaterial
    whether any two or more members attending it are in the same place
    as each other.
  \end{subclause}%
  \begin{subclause}%
    Two or more persons who are not in the same place as each other
    attend a general meeting if their circumstances are such that if
    they have (or were to have) rights to speak and vote at that
    meeting, they are (or would be) able to exercise them.
  \end{subclause}%
\end{clause}%
\begin{clause}{Quorum for general meetings}%
  \begin{subclause}%
    No business (other than the appointment of the chair of the
    meeting) may be transacted at any general meeting unless a quorum
    is present.
  \end{subclause}%
  \begin{subclause}%
    Two persons entitled to vote on the business to be transacted
    (each being a member, a proxy for a member or a duly Authorised
    Representative of a member); or 10\% of the total membership
    (represented in person or by proxy), whichever is greater, shall
    be a quorum.
  \end{subclause}%
  \begin{subclause}%
    If a quorum is not present within half an hour from the time
    appointed for the meeting, the meeting shall stand adjourned to
    the same day in the next week at the same time and place, or to
    such time and place as the Directors may determine, and if at the
    adjourned meeting a quorum is not present within half an hour from
    the time appointed for the meeting those present and entitled to
    vote shall be a quorum.
  \end{subclause}%
\end{clause}%
\begin{clause}{Chairing general meetings}%
  \label{cl:chairing}%
  \begin{subclause}%
    \label{cl:chair-presides}%
    The Chair (if any) or in his or her absence some other Director
    nominated by the Directors will preside as chair of every general
    meeting.
  \end{subclause}%
  \begin{subclause}%
    If neither the Chair nor such other Director nominated in
    accordance with Article \ref{cl:chair-presides} (if any) is
    present within fifteen minutes after the time appointed for
    holding the meeting and willing to act, the Directors present
    shall elect one of their number to chair the meeting and, if there
    is only one Director present and willing to act, he or she shall
    be chair of the meeting.
  \end{subclause}%
  \begin{subclause}%
    If no Director is willing to act as chair of the meeting, or if no
    Director is present within fifteen minutes after the time
    appointed for holding the meeting, the members present in person
    or by proxy and entitled to vote must choose one of their number
    to be chair of the meeting, save that a proxy holder who is not a
    member entitled to vote shall not be entitled to be appointed
    chair of the meeting.
  \end{subclause}%
\end{clause}%
\begin{clause}{Attendance and speaking by Directors and non-members}%
  \begin{subclause} 
    A Director may, even if not a member, attend and speak at any
    general meeting.
  \end{subclause}%
  \begin{subclause}%
    The chair of the meeting may permit other persons who are not
    members of the Company to attend and speak at a general meeting.
  \end{subclause}%
\end{clause}%
\begin{clause}{Adjournment}%
  \begin{subclause}%
    The chair of the meeting may adjourn a general meeting at which a
    quorum is present if:
    \begin{enumerate}[(a)]%
      \item the meeting consents to an adjournment; or
      \item it appears to the chair of the meeting that an adjournment
        is necessary to protect the safety of any person attending the
        meeting or ensure that the business of the meeting is
        conducted in an orderly manner.
    \end{enumerate}%
  \end{subclause}%
  \begin{subclause}%
    The chair of the meeting must adjourn a general meeting if
    directed to do so by the meeting.
  \end{subclause}%
  \begin{subclause}%
    When adjourning a general meeting, the chair of the meeting must:
    \begin{enumerate}[(a)]%
      \item either specify the time and place to which it is adjourned
        or state that it is to continue at a time and place to be
        fixed by the Directors; and
      \item have regard to any directions as to the time and place of
        any adjournment which have been given by the meeting.
    \end{enumerate}%
  \end{subclause}%
  \begin{subclause}%
    If the continuation of an adjourned meeting is to take place more
    than 14 days after it was adjourned, the Company must give at
    least seven Clear Days’ notice of it:
    \begin{enumerate}[(a)]%
      \item to the same persons to whom notice of the Company’s
        general meetings is required to be given; and
      \item containing the same information which such notice is
        required to contain.
    \end{enumerate}%
  \end{subclause}%
  \begin{subclause}%
    No business may be transacted at an adjourned general meeting
    which could not properly have been transacted at the meeting if
    the adjournment had not taken place.
  \end{subclause}%
\end{clause}%
\heading{Voting at General Meetings}%
\begin{clause}{Voting: general}%
  \begin{subclause}%
    A resolution put to the vote of a general meeting must be decided
    on a show of hands unless a poll is duly demanded in accordance
    with the Articles.
  \end{subclause}%
  \begin{subclause}%
    \label{cl:voting-nonmember}%
    A person who is not a member of the Company shall not have any
    right to vote at a general meeting of the Company; but this is
    without prejudice to any right to vote on a resolution affecting
    the rights attached to a class of the Company’s debentures.
  \end{subclause}%
  \begin{subclause}%
    Article \ref{cl:voting-nonmember} shall not prevent a person who
    is a proxy for a member or a duly Authorised Representative from
    voting at a general meeting of the Company.
  \end{subclause}%
\end{clause}%
\begin{clause}{Votes}%
  \label{cl:votes}%
  \begin{subclause}%
    On a vote on a resolution on a show of hands at a meeting every
    person present in person (whether a member, proxy or Authorised
    Representative of a member) and entitled to vote shall have a
    maximum of one vote.
  \end{subclause}%
  \begin{subclause}%
    On a vote on a resolution on a poll at a meeting every member
    present in person or by proxy or Authorised Representative shall
    have one vote.
  \end{subclause}%
  \begin{subclause}%
    In the case of an equality of votes, whether on a show of hands or
    on a poll, the chair of the meeting shall not be entitled to a
    casting vote in addition to any other vote he or she may have.
  \end{subclause}%
  \begin{subclause}%
    No member shall be entitled to vote at any general meeting unless
    all monies presently payable by him, her or it to the Company have
    been paid.
  \end{subclause}%
  \begin{subclause}%
    \label{cl:voting-orgrep}%
    The following provisions apply to any organisation that is a
    member (“a Member Organisation”):
    \begin{enumerate}[(a)]%
      \item a Member Organisation may nominate any individual to act
        as its representative (“an Authorised Representative”) at any
        meeting of the Company;
      \item the Member Organisation must give notice in Writing to the
        Company of the name of its Authorised Representative.  The
        Authorised Representative will not be entitled to represent
        the Member Organisation at any meeting of the Company unless
        such notice has been received by the Company.  The Authorised
        Representative may continue to represent the Member
        Organisation until notice in Writing is received by the
        Company to the contrary;
      \item a Member Organisation may appoint an Authorised
        Representative to represent it at a particular meeting of the
        Company or at all meetings of the Company until notice in
        Writing to the contrary is received by the Company;
      \item any notice in Writing received by the Company shall be
        conclusive evidence of the Authorised Representative’s
        authority to represent the Member Organisation or that his or
        her authority has been revoked.  The Company shall not be
        required to consider whether the Authorised Representative has
        been properly appointed by the Member Organisation;
      \item an individual appointed by a Member Organisation to act as
        its Authorised Representative is entitled to exercise (on
        behalf of the Member Organisation) the same powers as the
        Member Organisation could exercise if it were an individual
        member;
      \item on a vote on a resolution at a meeting of the Company, the
        Authorised Representative has the same voting rights as the
        Member Organisation would be entitled to if it was an
        individual member present in person at the meeting; and
      \item the power to appoint an Authorised Representative under
        this Article \ref{cl:voting-orgrep} is without prejudice to
        any rights which the Member Organisation has under the
        Companies Acts and the Articles to appoint a proxy or a
        corporate representative.
    \end{enumerate}%
  \end{subclause}%
\end{clause}%
\begin{clause}{Poll votes}%
  \begin{subclause}%
    A poll on a resolution may be demanded:
    \begin{enumerate}[(a)]%
      \item in advance of the general meeting where it is to be put to
        the vote; or
      \item at a general meeting, either before a show of hands on
        that resolution or immediately after the result of a show of
        hands on that resolution is declared.
    \end{enumerate}%
  \end{subclause}%
  \begin{subclause}%
    A poll may be demanded by:
    \begin{enumerate}[(a)]%
      \item the chair of the meeting;
      \item the Directors;
      \item two or more persons having the right to vote on the
        resolution;
      \item any person, who, by virtue of being appointed proxy for
        one or more members having the right to vote at the meeting,
        holds two or more votes; or
      \item a person or persons representing not less than one tenth
        of the total voting rights of all the members having the right
        to vote on the resolution.
    \end{enumerate}%
  \end{subclause}%
  \begin{subclause}%
    A demand for a poll may be withdrawn if:
    \begin{enumerate}[(a)]%
      \item the poll has not yet been taken; and
      \item the chair of the meeting consents to the withdrawal.
    \end{enumerate}%
  \end{subclause}%
  \begin{subclause}%
    Polls must be taken immediately and in such manner as the chair of
    the meeting directs.
  \end{subclause}%
\end{clause}%
\begin{clause}{Errors and disputes}%
  \begin{subclause}%
    No objection may be raised to the qualification of any person
    voting at a general meeting except at the meeting or adjourned
    meeting at which the vote objected to is tendered, and every vote
    not disallowed at the meeting is valid.
  \end{subclause}%
  \begin{subclause}%
    Any such objection must be referred to the chair of the meeting
    whose decision is final.
  \end{subclause}%
\end{clause}%
\begin{clause}{Content of proxy notices}%
  \label{cl:proxy-notices}%
  \begin{subclause}%
    Proxies may only validly be appointed by a notice in writing (a “Proxy
    Notice”) which:
    \begin{enumerate}[(a)]%
      \item states the name and address of the member appointing the
        proxy;
      \item identifies the person appointed to be that member’s proxy
        and the general meeting in relation to which that person is
        appointed;
      \item is signed by or on behalf of the member appointing the
        proxy, or is authenticated in such manner as the directors may
        determine; and
      \item is delivered to the Company in accordance with the
        Articles and any instructions contained in the notice of the
        general meeting to which they relate.
    \end{enumerate}%
  \end{subclause}%
  \begin{subclause}%
    The Company may require Proxy Notices to be delivered in a
    particular form, and may specify different forms for different
    purposes.
  \end{subclause}%
  \begin{subclause}%
    Proxy Notices may specify how the proxy appointed under them is to
    vote (or that the proxy is to abstain from voting) on one or more
    resolutions.
  \end{subclause}%
  \begin{subclause}%
    Unless a Proxy Notice indicates otherwise, it must be treated as:
    \begin{enumerate}[(a)]%
      \item allowing the person appointed under it as a proxy
        discretion as to how to vote on any ancillary or procedural
        resolutions put to the meeting; and
      \item appointing that person as a proxy in relation to any
        adjournment of the general meeting to which it relates as well
        as the meeting itself.
    \end{enumerate}%
  \end{subclause}%
\end{clause}%
\begin{clause}{Delivery of proxy notices}%
  \begin{subclause}%
    A person who is entitled to attend, speak or vote (either on a
    show of hands or on a poll) at a general meeting remains so
    entitled in respect of that meeting or any adjournment of it, even
    though a valid Proxy Notice has been delivered to the Company by
    or on behalf of that person.
  \end{subclause}%
  \begin{subclause}%
    An appointment under a Proxy Notice may be revoked by delivering
    to the Company a notice in Writing given by or on behalf of the
    person by whom or on whose behalf the Proxy Notice was given.
  \end{subclause}%
  \begin{subclause}%
    A notice revoking the appointment of a proxy only takes effect if
    it is delivered before the start of the meeting or adjourned
    meeting to which it relates.
  \end{subclause}%
\end{clause}%
\begin{clause}{Amendments to resolutions}%
  \begin{subclause}%
    An ordinary resolution to be proposed at a general meeting may be
    amended by ordinary resolution if:
    \begin{enumerate}[(a)]%
      \item notice of the proposed amendment is given to the Company
        in Writing by a person entitled to vote at the general meeting
        at which it is to be proposed not less than 48 hours before
        the meeting is to take place (or such later time as the chair
        of the meeting may determine); and
       \item the proposed amendment does not, in the reasonable
         opinion of the chair of the meeting, materially alter the
         scope of the resolution.
    \end{enumerate}%
  \end{subclause}%
  \begin{subclause}%
    A special resolution to be proposed at a general meeting may be
    amended by ordinary resolution, if:
    \begin{enumerate}[(a)]%
      \item the chair of the meeting proposes the amendment at the
        general meeting at which the resolution is to be proposed; and
      \item the amendment does not go beyond what is necessary to
        correct a grammatical or other non-substantive error in the
        resolution.
    \end{enumerate}%
  \end{subclause}%
  \begin{subclause}%
    If the chair of the meeting, acting in good faith, wrongly decides
    that an amendment to a resolution is out of order, the chair’s
    error does not invalidate the vote on that resolution.
  \end{subclause}%
\end{clause}%
\heading{Written Resolutions}%
\begin{clause}{Written Resolutions}%
  \label{cl:resolution}%
  \begin{subclause}%
    Subject to Article \ref{cl:resolution-removal}, a written
    resolution of the Company passed in accordance with this Article
    \ref{cl:resolution} shall have effect as if passed by the Company
    in general meeting:
    \begin{enumerate}[(a)]%
      \item A written resolution is passed as an ordinary resolution
        if it is passed by a simple majority of the total voting
        rights of eligible members.
      \item A written resolution is passed as a special resolution if
        it is passed by members representing not less than 75\% of the
        total voting rights of eligible members.  A written resolution
        is not a special resolution unless it states that it was
        proposed as a special resolution.
    \end{enumerate}%
  \end{subclause}%
  \begin{subclause}%
    In relation to a resolution proposed as a written resolution of
    the Company the eligible members are the members who would have
    been entitled to vote on the resolution on the circulation date of
    the resolution.
  \end{subclause}%
  \begin{subclause}%
    \label{cl:resolution-removal}%
    A members’ resolution under the Companies Acts removing a Director  or
    an auditor before the expiration of his or her term of office may  not
    be passed as a written resolution.
  \end{subclause}%
  \begin{subclause}%
    A copy of the written resolution must be sent to every member
    together with a statement informing the member how to signify
    their agreement to the resolution and the date by which the
    resolution must be passed if it is not to lapse.  Communications
    in relation to written notices shall be sent to the Company’s
    auditors in accordance with the Companies Acts.
  \end{subclause}%
  \begin{subclause}%
    A member signifies their agreement to a proposed written
    resolution when the Company receives from him or her an
    authenticated Document identifying the resolution to which it
    relates and indicating his or her agreement to the resolution.
    \begin{enumerate}[(a)]%
      \item If the Document is sent to the Company in Hard Copy Form,
        it is authenticated if it bears the member’s signature.
      \item If the Document is sent to the Company by Electronic
        Means, it is authenticated if the identity of the member is
        confirmed in a manner agreed by the Directors or if it is
        accompanied by a statement of the identity of the member and
        the Company has no reason to doubt the truth of that
        statement.
    \end{enumerate}%
  \end{subclause}%
  \begin{subclause}%
    A written resolution is passed when the required majority of
    eligible members have signified their agreement to it.
  \end{subclause}%
  \begin{subclause}%
    A proposed written resolution lapses if it is not passed within 28
    days beginning with the circulation date.
  \end{subclause}%
\end{clause}%
\newpage%
\heading{Administrative Arrangements and Miscellaneous}%
\begin{clause}{Means of communication to be used}%
  \begin{subclause}%
    Subject to the Articles, anything sent or supplied by or to the
    Company under the Articles may be sent or supplied in any way in
    which the Companies Act 2006 provides for Documents or information
    which are authorised or required by any provision of that Act to
    be sent or supplied by or to the Company.
  \end{subclause}%
  \begin{subclause}%
    Subject to the Articles, any notice or Document to be sent or
    supplied to a Director in connection with the taking of decisions
    by Directors may also be sent or supplied by the means by which
    that Director has asked to be sent or supplied with such notices
    or Documents for the time being.
  \end{subclause}%
  \begin{subclause}%
    A Director may agree with the Company that notices or Documents
    sent to that Director in a particular way are to be deemed to have
    been received within an agreed time of their being sent, and for
    the agreed time to be less than 48 hours.
  \end{subclause}%
\end{clause}%
\begin{clause}{Irregularities}%
  The proceedings at any meeting or on the taking of any poll or the
  passing of a written resolution or the making of any decision shall
  not be invalidated by reason of any accidental informality or
  irregularity (including any accidental omission to give or any non-
  receipt of notice) or any want of qualification in any of the
  persons present or voting or by reason of any business being
  considered which is not referred to in the notice unless a provision
  of the Companies Acts specifies that such informality, irregularity
  or want of qualification shall invalidate it.
\end{clause}%
\begin{clause}{Minutes}%
  \label{cl:minutes}%
  \begin{subclause}%
    The Directors must cause minutes to be made in books kept for the
    purpose:
    \begin{enumerate}[(a)]%
      \item of all appointments of officers made by the Directors;
      \item of all resolutions of the Company and of the Directors; and
      \item of all proceedings at meetings of the Company and of the
        Directors, and of committees of Directors, including the names
        of the Directors present at each such meeting;
    \end{enumerate}%
    and any such minute, if purported to be signed (or in the case of
    minutes of Directors’ meetings signed or authenticated) by the
    chair of the meeting at which the proceedings were had, or by the
    chair of the next succeeding meeting, shall, as against any member
    or Director of the Company, be sufficient evidence of the
    proceedings.
  \end{subclause}%
  \begin{subclause}%
     The minutes must be kept for at least ten years from the date of
     the meeting, resolution or decision.
  \end{subclause}%
\end{clause}%
\begin{clause}{Records and accounts}%
  The Directors shall comply with the requirements of the Companies
  Acts as to maintaining a members’ register, keeping financial
  records, the audit or examination of accounts and the preparation
  and transmission to the Registrar of Companies and the Regulator of:
  \begin{enumerate}[(a)]%
    \item annual reports;
    \item annual returns; and
    \item annual statements of account.
  \end{enumerate}%
\end{clause}%
\begin{clause}{Indemnity}%
  \begin{subclause}%
    Subject to Article \ref{cl:indemnity-void}, a relevant Director of
    the Company or an associated company may be indemnified out of the
    Company’s assets against:
    \begin{enumerate}[(a)]%
      \item any liability incurred by that Director in connection with
        any negligence, default, breach of duty or breach of trust in
        relation to the Company or an associated company;
      \item any liability incurred by that Director in connection  with  the
        activities of the  Company  or  an  associated  company  in  its
        capacity as a trustee of  an  occupational  pension  scheme  (as
        defined in section 235(6) of the Companies Act 2006); and
      \item any other liability incurred by that Director as an  officer  of
        the Company or an associated company.
    \end{enumerate}%
  \end{subclause}%
  \begin{subclause}%
    \label{cl:indemnity-void}%
    This  Article  does  not  authorise  any  indemnity  which  would  be
    prohibited or rendered void by any provision of the Companies Acts  or
    by any other provision of law.
  \end{subclause}%
  \begin{subclause}%
    In this Article:
    \begin{enumerate}[(a)]%
      \item companies are associated if one is a subsidiary of the other  or
        both are subsidiaries of the same body corporate; and
      \item  a “relevant Director” means any Director or former  Director  of
        the Company or an associated company.
    \end{enumerate}%
  \end{subclause}%
\end{clause}%
\begin{clause}{Insurance}%
  \begin{subclause}%
    The Directors may decide to purchase and maintain insurance, at
    the expense of the Company, for the benefit of any relevant
    Director in respect of any relevant loss.
  \end{subclause}%
  \begin{subclause}%
    In this Article:
    \begin{enumerate}[(a)]%
      \item a “relevant Director” means any Director or former
        Director of the Company or an associated company;
      \item a “relevant loss” means any loss or liability which has
        been or may be incurred by a relevant Director in connection
        with that Director’s duties or powers in relation to the
        Company, any associated company or any pension fund or
        employees’ share scheme of the company or associated
        company; and
      \item companies are associated if one is a subsidiary of the
        other or both are subsidiaries of the same body corporate.
    \end{enumerate}%
  \end{subclause}%
\end{clause}%
\begin{clause}{Exclusion of model articles}%
  The relevant model articles for a company limited by guarantee are
  hereby expressly excluded.
\end{clause}%
\newpage%

\schedule{Interpretation}%

\begin{clause}{Defined terms}%
  In the Articles, unless the context requires otherwise, the
  following terms shall have the following meanings:
  \par%
  \vspace{\baselineskip}%
  \begin{terms}%
    \term{``Address''}{includes a number or address used for the purposes
      of sending or receiving Documents by Electronic Means;}%
    \term{``Articles''}{the Company’s articles of association;}%
    \term{``Authorised Representative''}{means any individual nominated by
      a Member Organisation to act as its representative at any
      meeting of the Company in accordance with Article
      \ref{cl:votes};}%
    \term{``asset-locked body''}{means
      \begin{enumerate}[(i)]
        \item a community interest company, a charity or a Permitted
          Industrial and Provident Society; or
        \item a body established outside the United Kingdom that is
          equivalent to any of those;
      \end{enumerate}}%
    \term{``bankruptcy''}{includes individual insolvency proceedings in a
      jurisdiction other than England and Wales or Northern Ireland
      which have an effect similar to that of bankruptcy;}%
    \term{``Chair''}{has the meaning given in Article \ref{cl:chair};}%
    \term{``chairman of the meeting''}{has the meaning given in Article
      \ref{cl:chairing}; }%
    \term{``Circulation Date''}{in relation to a written resolution, has
      the meaning given to it in the Companies Acts;}%
    \term{``Clear Days''}{in relation to the period of a notice, that
      period excluding the day when the notice is given or deemed to
      be given and the day for which it is given or on which it is to
      take effect; }%
    \term{``community''}{is to be construed in accordance with accordance
      with Section 35(5) of the Company’s (Audit) Investigations and
      Community Enterprise) Act 2004;}%
    \term{``Companies Acts''}{means the Companies Acts (as defined in
      Section 2 of the Companies Act 2006), in so far as they apply to
      the Company;}%
    \term{``Company''}{\companyname;}%
    \term{``Conflict of Interest''}{any direct or indirect interest of a
      Director (whether personal, by virtue of a duty of loyalty to
      another organisation or otherwise) that conflicts, or might
      conflict with the interests of the Company;}%
    \term{``Director''}{a director of the Company, and includes any person
      occupying the position of director, by whatever name called;}%
    \term{``Document''}{includes, unless otherwise indicated, any Document
      sent or supplied in Electronic Form; }%
    \term{``Electronic Form'' and ``Electronic Means''''}{have the
      meanings respectively given to them in Section 1168 of the
      Companies Act 2006;}%
    \term{``Hard Copy Form''}{has the meaning given to it in the Companies
      Act 2006;}%
    \term{``Memorandum''}{the Company’s memorandum of association;}%
    \term{``paid''}{means paid or credited as paid;}%
    \term{``participate''}{in relation to a Directors’ meeting, has the
      meaning given in Article \ref{cl:participation};}%
    \term{``Permitted Industrial and Provident Society''}{an industrial
      and provident society which has a restriction on the use of its
      assets in accordance with Regulation 4 of the Community Benefit
      Societies (Restriction on Use of Assets) Regulations 2006 or
      Regulation 4 of the Community Benefit Societies (Restriction on
      Use of Assets) Regulations (Northern Ireland) 2006;}%
    \term{``Proxy Notice''}{has the meaning given in Article
      \ref{cl:proxy-notices};  } 
    \term{``the Regulator''}{means the Regulator of Community Interest
      Companies;}%
    \term{``Secretary''}{the secretary of the Company (if any);}%
    \term{``specified''}{means specified in the memorandum and articles of
      association of the Company for the purposes of this paragraph;}%
    \term{``subsidiary''}{has the meaning given in section 1159 of the
      Companies Act 2006;}%
    \term{``transfer''}{includes every description of disposition,
      payment, release or distribution, and the creation or extinction
      of an estate or interest in, or right over, any property; and}%
    \term{``Writing''}{the representation or reproduction of words,
      symbols or other information in a visible form by any method or
      combination of methods, whether sent or supplied in Electronic
      Form or otherwise.}%
  \end{terms}%
\end{clause}%
\begin{clause*}%
  Subject to clause \ref{cl:definitions-companies-act} of this
  Schedule, any reference in the Articles to an enactment includes a
  reference to that enactment as re-enacted or amended from time to
  time and to any subordinate legislation made under it.
\end{clause*}%
\begin{clause*}%
  \label{cl:definitions-companies-act}%
  Unless the context otherwise requires, other words or expressions
  contained in these Articles bear the same meaning as in the
  Companies Act 2006 as in force on the date when the Articles become
  binding on the Company.
\end{clause*}%
\end{document}%
